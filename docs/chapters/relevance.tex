\section{Актуальность}\label{sec:}

Польза от системы дистанционного наблюдения за пациентами была проанализирована и описана в статье \("\)Study design of Heart failure Events reduction with Remote Monitoring and eHealth Support (HERMeS)\("\)\, 2020. \par
Люди, подходящие под определенные критерии, были случайным образом разделены на две группы в соотношении 1:1.
Все пациенты, включенные в исследование, находились под наблюдением в течение 6 месяцев.
Пациенты, включенные в группу ТМ (Telemedicine group), находились под дистанционным наблюдением и наблюдением в соответствии с конкретным клиническим маршрутом, который включает заранее запланированные структурированные последующие контакты с командой здравоохранения, использующей VC. Пациенты в отделении UC наблюдались в соответствии с UC каждого рекрутингового центра. \par
Наблюдение и лечение пациентов в отделении ТМ будут основаны на платформе PIRENe. Платформа PIRENe представляет собой комплексное решение для ухода и мониторинга хронических пациентов, смоделированное и протестирован на пациентах с хронической СН.
Эта платформа позволяет предоставлять многоканальное обслуживание и мониторинг пациентов посредством: \par
\begin{enumerate}
    \itemМониторинга пациента
    \begin{enumerate}
        \item Биометрические данные (вес, ЧСС и АД);
        \item Отчет о симптомах: семь вопросов, чтобы отразить ухудшение симптомов сердечно-сосудистых заболеваний, в основном ухудшение СН, и один вопрос, чтобы отразить общее ухудшение (см.\ Таблицу~\ref{tab:timesandtenses}).
        Вопросы ставятся так, чтобы ответить «да» или «нет».
    \end{enumerate}
    \item Генерации и управления предупреждающими сигналами, отправляемыми специалистам, закрепленным за каждым пациентом, в случае возникновения одной из следующих ситуаций:
    \begin{enumerate}
        \item Биометрические данные за пределами допустимого диапазона.
        \item Любой тревожный симптом среди ответов на анкету (один ответ «да» в анкете генерирует предупреждающий сигнал в платформе).
        \item Отсутствие биометрических измерений в любой день
    \end{enumerate}
    \item Последующее наблюдение посредством телеконсультаций (видеоконференция, аудиоконференция, рассылка по почте и управление входящими и исходящими звонками) между специалистами и пациентами/опекунами.
\end{enumerate}

\begin{table}[h]
    \caption{Вопросы, на которые отвечали пациенты.}
    \label{tab:timesandtenses}
    \begin{center}
        \begin{tabular}{|V{10cm}|c|}
            \hline
            Вопрос & Ответ \\
            \hline
            Мои ноги опухли больше, чем обычно & Да/Нет \\
            \hline
            Я чувствую себя более утомленным, уставшим или с ощущением удушья.
            & Да/Нет \\
            \hline
            У плохо спал из-за одышки или ощущения удушья.
            & Да/Нет \\
            \hline
            Мне нужно было больше подушек, чтобы лучше дышать ночью, лежа на кровати.
            & Да/Нет \\
            \hline
            Мне приходилось спать сидя из-за одышки или ощущения удушья, когда я лежал на кровати.
            & Да/Нет \\
            \hline
            Я чувствовал слабость и головокружение сильнее, чем обычно.
            & Да/Нет \\
            \hline
            У меня сильнее болела грудь, чем обычно.
            & Да/Нет \\
            \hline
            В общем, я чувствую себя хуже, чем обычно.
            & Да/Нет \\
            \hline
        \end{tabular}
    \end{center}
\end{table}

Проведенные исследования показали, что данная стратегия предоставления управляемой помощи, приведет к значительному снижению смертности или повторной госпитализации у этих пациентов высокого риска в \("\)уязвимой фазе\("\) заболевания.
